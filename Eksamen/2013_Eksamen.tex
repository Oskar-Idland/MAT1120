\documentclass{article}
\usepackage{amsmath}
\usepackage[mathletters]{ucs}
\usepackage[utf8x]{inputenc}
\usepackage[margin=1.5in]{geometry}
\usepackage{enumerate}
\newtheorem{theorem}{Theorem}
\usepackage[dvipsnames]{xcolor}
\usepackage{pgfplots}
\setlength{\parindent}{0cm}
\usepackage{graphics}
\usepackage{graphicx} % Required for including images
\usepackage{subcaption}
\usepackage{bigintcalc}
\usepackage{pythonhighlight} %for pythonkode \begin{python}   \end{python}
\usepackage{appendix}
\usepackage{arydshln}
\usepackage{physics}
\usepackage{tikz-cd}
\usepackage{booktabs} 
\usepackage{adjustbox}
\usepackage{mdframed}
\usepackage{relsize}
\usepackage{physics}
\usepackage[thinc]{esdiff}
\usepackage{fixltx2e}
\usepackage{esint}  %for lukket-linje-integral
\usepackage{xfrac} %for sfrac
\usepackage{hyperref} %for linker, må ha med hypersetup
\usepackage[noabbrev, nameinlink]{cleveref} % to be loaded after hyperref
\usepackage{amssymb} %\mathbb{R} for reelle tall, \mathcal{B} for "matte"-font
\usepackage{listings} %for kode/lstlisting
\usepackage{verbatim}
\usepackage{graphicx,wrapfig,lipsum,caption} %for wrapping av bilder
\usepackage{mathtools} %for \abs{x}
\usepackage[norsk]{babel}
\definecolor{codegreen}{rgb}{0,0.6,0}
\definecolor{codegray}{rgb}{0.5,0.5,0.5}
\definecolor{codepurple}{rgb}{0.58,0,0.82}
\definecolor{backcolour}{rgb}{0.95,0.95,0.92}

\lstdefinestyle{mystyle}{
    backgroundcolor=\color{backcolour},   
    commentstyle=\color{codegreen},
    keywordstyle=\color{magenta},
    numberstyle=\tiny\color{codegray},
    stringstyle=\color{codepurple},
    basicstyle=\ttfamily\footnotesize,
    breakatwhitespace=false,         
    breaklines=true,                 
    captionpos=b,                    
    keepspaces=true,                 
    numbers=left,                    
    numbersep=5pt,                  
    showspaces=false,                
    showstringspaces=false,
    showtabs=false,                  
    tabsize=2
}

\lstset{style=mystyle}
\author{Oskar Idland}
\title{MAT1120 Eksamen 2013}
\date{}
\begin{document}
\maketitle
\newpage

\section*{Oppgave 1}
\subsection*{a)}

\[
A = 
\begin{pmatrix}
 1 & 1 & 0 & 2 \\
 1 & -1 & 1 & 1 \\
 1 & 0 & 0 & 1 \\
 0 & 1 & 1 & 2 \\
\end{pmatrix} \xrightarrow{rref}

\quad

C  =  \operatorname{rref}(A) =
\begin{pmatrix*}[r]
 1 & 0 & 0 & 1 \\
 0 & 1 & 0 & 1 \\
 0 & 0 & 1 & 1 \\
 0 & 0 & 0 & 0 \\
\end{pmatrix*}
\]

From C we know that only the three first columns in $A$ are linearly independent which means its rank is 3. The Null space must therefore be 1-dimensional. To find the basis for the null space we can use the following equation:
\[
A \mathbf{x} = \mathbf 0
\]
With the help of matrix $C$ we can solve this equation easily
\[
\begin{pmatrix*}[r]
 1 & 0 & 0 & 1 \\
 0 & 1 & 0 & 1 \\
 0 & 0 & 1 & 1 \\
 0 & 0 & 0 & 0 \\
\end{pmatrix*}
\begin{pmatrix*}[r]
 x_1 \\
 x_2 \\
 x_3 \\
 x_4 \\
\end{pmatrix*} = 
\begin{pmatrix*}[r]
 0 \\
 0 \\
 0 \\
 0 \\
\end{pmatrix*} = 
x_4 
\begin{pmatrix*}[r]
 -1 \\
 -1 \\
 -1 \\
 1 \\
\end{pmatrix*}
\]
The basis of the Null space is therefore 
\[
\left\{ 
    \begin{pmatrix*}[r]
    1 \\
    1 \\
    1 \\
    0 \\
    \end{pmatrix*} 
\right\} 
\]

\subsection*{b)}
To find and orthogonal basis of the Column space we can use the Gram-Schmidt process. We start by finding the first vector in the basis. This is the first column in $A$ which is
\[
\mathbf{v_1} = \mathbf{x_1} = 
\begin{pmatrix*}[r]
    1 \\
    1 \\
    1 \\
    0 \\
\end{pmatrix*}
\]

\[
\mathbf{v_2} = \mathbf{x_2} - \operatorname{proj}_{\mathbf{x_2}}(\mathbf{v_1}) = \mathbf{x_2} - \frac{\mathbf{x_2} \cdot \mathbf{v_1}}{\mathbf{v_1}, \mathbf{v_1}} \mathbf{v_1} = \mathbf{x_2} - \frac{0}{3}\mathbf{v_1} = \mathbf{x_2} = \begin{pmatrix*}[r]
 1 \\
 -1 \\
 0 \\
 1 \\
\end{pmatrix*}
\]
\[
\mathbf{v_3} = \mathbf{x_3} - \operatorname{proj}_{\mathbf{x_3}}(\mathbf{v_1}) - \operatorname{proj}_{\mathbf{x_3}}(\mathbf{v_2}) = \mathbf{x_3} - \frac{\mathbf{x_3} \cdot \mathbf{v_1}}{\mathbf{v_1} \cdot \mathbf{v_1}} \mathbf{v_1} - \frac{\mathbf{x_3} \cdot \mathbf{v_2}}{\mathbf{v_2} \cdot \mathbf{v_2}} \mathbf{v_2} = \mathbf{x_3} - \frac{1}{3}\begin{pmatrix*}[r]
 1 \\
 1 \\
 1 \\
 0 \\
\end{pmatrix*}  - 
\frac{0}{3} \begin{pmatrix*}[r]
 1 \\
 -1 \\
 0 \\
 1 \\
\end{pmatrix*}
\]
\[
\mathbf{v_3} = \begin{pmatrix*}[r]
 0 \\
 1 \\
 0 \\
 1 \\
\end{pmatrix*} - 
\frac{1}{3}
\begin{pmatrix*}[r]
 1 \\
 1 \\
 1 \\
 0 \\
\end{pmatrix*} = 
\begin{pmatrix*}[r]
 -\frac{1}{3} \\
 \frac{2}{3} \\
 -\frac{1}{3} \\
 1 \\
\end{pmatrix*}
\]
\[
\mathbf{v_3}' = 3\mathbf{v_3} = 
\begin{pmatrix*}[r]
 -1 \\
 2 \\
 -1 \\
 3 \\
\end{pmatrix*}
\]

\[
\mathbf{v_4} = \mathbf{x_4} - \operatorname{proj}_{\mathbf{x_4}}(\mathbf{v_1}) - \operatorname{proj}_{\mathbf{x_4}}(\mathbf{v_2}) - \operatorname{proj}_{\mathbf{x_4}}(\mathbf{v_3}')  
\]
\[
\mathbf{v_4} = \mathbf{x_4} - \frac{\mathbf{x_4} \cdot \mathbf{v_1}}{\mathbf{v_1} \cdot \mathbf{v_1}} \mathbf{v_1} - \frac{\mathbf{x_4} \cdot \mathbf{v_2}}{\mathbf{v_2} \cdot \mathbf{v_2}} \mathbf{v_2} - \frac{\mathbf{x_4} \cdot \mathbf{v_3}'}{\mathbf{v_3}' \cdot \mathbf{v_3}'} \mathbf{v_3}' 
\]

\[
\mathbf{v_4} = \mathbf{x_4} - \frac{4}{3}\begin{pmatrix*}[r]
 1 \\
 1 \\
 1 \\
 0 \\
\end{pmatrix*} - 
\frac{3}{3}\begin{pmatrix*}[r]
 1 \\
 -1 \\
 0 \\
 1 \\
\end{pmatrix*} -
\frac{5}{15}\begin{pmatrix*}[r]
 -1 \\
 2 \\
 -1 \\
 3 \\
\end{pmatrix*}
\]

\[
\mathbf{v_4} = \mathbf{x_4} - \frac{1}{3}\begin{pmatrix*}[r]
 -4 - 1 + 1 \\
 -4 + 1 - 2 \\
 -4 + 0 + 1 \\
 0 - 1 - 3 \\
\end{pmatrix*} = 
\begin{pmatrix*}[r]
 2 \\
 1 \\
 1 \\
 2 \\
\end{pmatrix*}
 - \frac{1}{3}
 \begin{pmatrix*}[r]
  -4 \\
  -5 \\
  -3 \\
  -3 \\
 \end{pmatrix*}
\]

\[
\mathbf{v_4} = 
\frac{1}{3}
\begin{pmatrix*}[r]
 6 + 4 \\
 3 + 5  \\
 3 + 3 \\
 6 + 3 \\
\end{pmatrix*}
\]

\[
\mathbf{v_4} = \frac{1}{3}
\begin{pmatrix*}[r]
 10 \\
 8 \\
 5 \\
 9 \\
\end{pmatrix*}
\]




\end{document}